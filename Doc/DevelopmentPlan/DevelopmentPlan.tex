\documentclass{article}

\usepackage{booktabs}
\usepackage{tabularx}
\usepackage{hyperref}

\title{SE 3XA3: Development Plan\\ReTouch}

\author{Team \#7, ReTouchers
		\\ Abrar Attia - attiaa1
		\\ Susan Fayez - fayezs
		\\ Mediha Munim - munimm
}

\date{September 29, 2017}

\begin{document}

\begin{table}[hp]
\caption{Revision History} \label{TblRevisionHistory}
\begin{tabularx}{\textwidth}{llX}
\toprule
\textbf{Date} & \textbf{Developer(s)} & \textbf{Change}\\
\midrule
Sept 27 & Mediha Munim & Updated template with project/team details\\
Sept 28 & Mediha Munim & Completed intro - Git worflow plan\\
Sept 28 & Abrar Attia & Completed POC - Coding Style\\
Sept 29 & Mediha Munim & Made small edits, added GanttProject link\\
... & ... & ...\\
\bottomrule
\end{tabularx}
\end{table}

\newpage

\maketitle

KTouch is an open source program that was designed to improve the typing abilities of its users. Reimplementing the program as ReTouch will present several challenges, and this document provides an efficient plan to produce the software with ease.

\section{Team Meeting Plan}

	The ReTouchers team has agreed upon a weekly meeting schedule to ensure the project stays on schedule and all milestones are completed on time. On top of meeting twice a week during the labs on Mondays and Thursdays, the team plans to meet from 11:30 a.m. - 12:20 p.m. on Wednesdays in a study room in Thode Library. If the allotted time is insufficient, the group will decide on another convenient time to meet. 

The notes from each meeting should follow an organized format. They will consist of an "Agenda" section (describing what needs to be discussed in the meeting), a "Minutes" section, and end with a "Homework" section (which will detail what needs to be done).

\section{Team Communication Plan}

	The team plans to communicate primarily through Facebook Messenger since it is easy to use and is checked regularly by all members of ReTouchers. Discussions on Facebook will be about general questions and for decision making. Git issues will also be used to further ensure proper communication by allowing another way for team members to clarify actions/items on Gitlab. 

\section{Team Member Roles}

	The team roles were assigned early on and are subject to change in the future. Currently, Abrar is the team leader, the front end developer, and the Git expert. Susan is the C++/C expert and one of the main back-end Java coders. Mediha is the second main back-end Java coder, and is also the Latex/documentation expert. Development of the actual implementation will therefore be split equally amongst team members, with each member focusing on their main area of expertise.

	In meetings, the meeting chair will also be the scribe. This role will rotate between the team members every week.

\section{Git Workflow Plan}

	The Git workflow plan that the ReTouchers team has decided to follow is the "Feature Branch Workflow." This workflow takes advantage of branching - every time a team member wants to add a 
certain feature, they can create a\textit{ feature branch} and commit their updated files there. They can push their changes to the central repository without affecting the master branch. Then, a pull request needs to be filed so that the other team members can verify the changes and merge the feature branch with the master branch. Labels will be useful since they can be added to merge requests and issues, as well as when a bug has been discovered. It has also been agreed upon that milestones will be used to track the project's main deliverables.

\section{Proof of Concept Demonstration Plan}

    The most significant risks that exist with developing the ReTouch application begin with the expected difficulties during implementation. The original KTouch application is written in C/C++, and there may be some difficulties transferring the code structure and functionality into Java. In addition, many of the required libraries in the original application may not be portable and therefore may not work with Java code. In order to overcome this risk, similar Java libraries that resemble and function like the C/C++ libraries may be found and used for the ReTouch application. If no libraries are found, a wrapper can be used around the C/C++ libraries to make them usable in Java. Another risk with developing the application is that implementing a full GUI may be difficult and time consuming. Additionally, testing all the functionalities of the application may also cause some trouble. Furthermore, to overcome these risks, tasks will be divided amongst the developers to make sure that testing and GUI implementation are being done successfully and on schedule.


\section{Technology}

    The technology that will be used for the implementation, testing, and documentation of the ReTouch project is as follows. The main programming language that will be used to implement the project is Java. The IDE's that will be used are Eclipse or Net Beans depending on the developer's preference. In order to test the application, JUnit will be used for standard unit testing. Lastly, the documentation will be broken down into Latex and Doxygen for project documentation, and JavaDocs for code documentation.

\section{Coding Style}

    The coding style that best suits the ReTouch project implementation follows the Google Java Guide. The source code to be written will follow all of the structure, syntax rules, naming, and alignment that is provided in the Google Java Guide. 

\section{Project Schedule}

The Gantt Project, which illustrates the project schedule, can be found 
\href{https://gitlab.cas.mcmaster.ca/munimm/3XA3_Project/blob/master/ProjectSchedule/Gantt_Project_Revision_0.gan}{here} or in the ProjectSchedule directory. 


\section{Project Review}

\end{document}
