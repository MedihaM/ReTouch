\documentclass[12pt, oneside]{article}   
\usepackage{geometry}                		
\geometry{letterpaper}       
\usepackage{graphicx}				
\usepackage{float} 
\usepackage{amssymb}
\usepackage{color}
\usepackage{hyperref}
\usepackage{booktabs}
\usepackage{tabularx}
\usepackage[T1]{fontenc}
\usepackage{titling}
\usepackage{color}

\setlength{\droptitle}{-6em}

\hypersetup{
    colorlinks,
    citecolor=black,
    filecolor=black,
    linkcolor=black,
    urlcolor=black
}

\usepackage{fancyhdr}
\usepackage{fancyhdr}
\fancyhead[L]{December 6, 2017}
\fancyhead[C]{SE 3XA3 Problem Statement}
\fancyhead[R]{Group 7}
\pagestyle{fancy}

\usepackage{float}

%SetFonts

%SetFonts
		
\title{SE 3XA3: Problem Statement\\ReTouch}

\author{Group 7
		\\ Mediha Munim; munimm
		\\ Abrar Attia; attiaa1
		\\ Susan Fayez; fayezs}
		
\date{December 6, 2017}				
%---------------------------------------------------------------------
\begin{document}

\begin{table}[!h]
\caption{Revision History} \label{TblRevisionHistory}
\begin{tabularx}{\textwidth}{llX}
\toprule
\textbf{Date} & \textbf{Developer(s)} & \textbf{Change}\\
\midrule
Sept 21 & Susan, Abrar, Mediha & Discussed and wrote out rough copy\\
Sept 24 & Abrar & Typed out initial structure of problem statement in Latex\\
Sept 24 & Mediha & Edited paragraphs and overall document format \\
Dec 4 & Mediha & Removed scope, added headings, improved stakeholder and envronment descriptions\\
\bottomrule
\end{tabularx}
\end{table}

\newpage
\maketitle

%---------------------------------------------------------------------

%---------------------------------------------------------------------
%\maketitle
\color{cyan}
\section{What problem are you trying to solve?}
\color{black}

In a world so dependent on and advanced by technology, it has became a requirement for individuals to be proficient with a computer. This includes being able to perform many tasks such as writing up documents, recording data and communicating through virtual methods. In order to complete these tasks efficiently, people in a variety of fields are expected to be able to type quickly and accurately. Therefore, the goal for this project is to develop an application that will assist users in improving their typing skills and allow them to become more proficient with computer usage. 
\\
\color{cyan}
\section{Why is this an important problem?}
\color{black}

Improving typing speed and accuracy is very important for several reasons. To begin, the average employer expects their employees to have basic computer skills, including typing. Furthermore, individuals more often than not are expected to have at minimum average typing skills. In addition, learning to type quickly and accurately saves time and consequently saves money. Also, a large focus in today's world is improving proficiency at work and at home, and learning to type better is one very easy solution to this. Lastly, by becoming proficient at typing, individuals may have an advantage over other candidates when applying for jobs and, therefore, may open up more opportunities for themselves.
\\

\color{cyan}
\subsection{Who are the stakeholders?}
\color{black}

The stakeholders for this application include the current developers (Group 7) and any future developers who may contribute to the project. \textcolor{cyan}{These developers are responsible for the quality and functionalities of the software}. The software clients are Dr.Bokhari and the course TA's, \textcolor{cyan}{who will oversee the project's development and provide feedback for its improvement}. Lastly, the final stakeholders are all of the potential ReTouch users who are looking to improve their typing speed and accuracy. 
\\
\color{cyan} \subsection{What is the environment for the software?}
\color{black}

ReTouch will be an open-source software that is meant to assist individuals in improving their typing skills. Therefore, it will mostly be used in classrooms and other educational institutions, but it will be available to anyone else seeking to improve their typing skills independently \color{cyan}on their personal desktops. It will be accessible on computers as an application, and it should be portable across Windows, Linux, and Mac computers.

\end{document}  

