\documentclass[12pt, titlepage]{article}

\usepackage{booktabs}
\usepackage{tabularx}
\usepackage{hyperref}
\usepackage{longtable}
\usepackage{array}

\hypersetup{
    colorlinks,
    citecolor=black,
    filecolor=black,
    linkcolor=red,
    urlcolor=blue
}
\usepackage[round]{natbib}

\title{SE 3XA3: Software Requirements Specification\\ReTouch}

\author{Team \#7, ReTouchers
		\\ Abrar Attia - attiaa1
		\\ Susan Fayez - fayezs
		\\ Mediha Munim - munimm
}

\date{\today}

\begin{document}

\maketitle

\pagenumbering{roman}
\tableofcontents
\listoftables
\listoffigures

\begin{table}[bp]
\caption{\bf Revision History}
\begin{tabularx}{\textwidth}{p{3cm}p{2cm}X}
\toprule {\bf Date} & {\bf Version} & {\bf Notes}\\
\midrule
Oct 4 & 1.0 & Updated template with project/team names\\
Date 2 & 1.1 & Notes\\
\bottomrule
\end{tabularx}
\end{table}

\newpage

\pagenumbering{arabic}

This document describes the requirements for ....  The template for the Software
Requirements Specification (SRS) is a subset of the Volere
template~\citep{RobertsonAndRobertson2012}.  If you make further modifications
to the template, you should explicity state what modifications were made.

\section{Project Drivers}

\subsection{The Purpose of the Project}

\subsection{The Stakeholders}

\subsubsection{The Client}

\subsubsection{The Customers}

\subsubsection{Other Stakeholders}

\subsection{Mandated Constraints}

\subsection{Naming Conventions and Terminology}

\begin{itemize}
    \item Completed character: A character is completed if the user has successfully typed it on the keyboard when it was the current character. 
    \item Current character: The character that the user is expected to input at a given moment.
    \item Incorrect character: A character is incorrect if the wrong keyboard character was inputted by the user when it had been the current character.
    \item Noncompleted character: A character is noncompleted if it is not a completed character.
    \item Lesson beginning: The moment when the list of words is first generated and displayed on screen.
    \item Lesson end: The moment when the last character in the current list of words has been completed.
\end{itemize}



\subsection{Relevant Facts and Assumptions}

User characteristics should go under assumptions.

\section{Functional Requirements}

\subsection{The Scope of the Work and the Product}

\subsubsection{The Context of the Work}

\subsubsection{Work Partitioning}

\subsubsection{Individual Product Use Cases}

\subsection{Functional Requirements}



\begin{center}
\begin{longtable}{ |m{2cm}|m{1.8cm}|m{9.4cm}| } 
    \hline
    \textbf{Identifier} & \textbf{Priority} & \textbf{Requirement} \\ 
    \hline
    FREQ1 & 2 & The system should allow the user to choose between various lessons, each lesson onsisting of a different combination of keyboard symbols/letters. \\ 
    \hline
    FREQ2 & 5 & The system shall generate a list of characters consisting of the specified letter/symbol combinations. The list shall consist of around 500-600 characters, including spaces. \\ 
    \hline
    FREQ3 & 5 & The system shall display the list of characters using the specified character combinations for the user to type up. The list of characters will be presented on separate lines, with each line being no greater than a predetermined length.  \\ 
    \hline
    FREQ4 & 5 & The system shall begin the program at the first character (which will become the current character) and wait for the user to type a character. \\ 
    \hline
    FREQ5 & 5 & The system shall move the current character to the next character (on the right) when the user inputs a character (incorrect or not).  \\ 
    \hline
    FREQ6 & 5 & The system shall set a current character as “correct” if the user presses the same character that is indicated by the current character. \\ 
    \hline
    FREQ7 & 4 & The system shall indicate when an incorrect character has been typed by highlighting the incorrect character. All characters typed after an incorrect character will be considered incorrect, and highlighted as well. \\ 
    \hline
    FREQ8 & 4 & The system shall allow the user to move on to the next line only when they have reached the last character on the current line and all the characters on the line are correct. When the “ENTER” key is pressed, the current character will become the first character on the next line. \\ 
    \hline
    FREQ9 & 4 & The system shall remove the character to the left of the current character and move the current character to the left when the “BACKSPACE” key is pressed. However, if the current character is the first character of a line, nothing will happen when “BACKSPACE” is pressed. \\ 
    \hline
    FREQ10 & 2 & The system should count the number of times an incorrect character is typed. \\ 
    \hline
    FREQ11 & 2 & The system should display the elapsed time from when a lesson begins to when the lesson is completed. \\ 
    \hline
    FREQ12 & 2 & The system should display the typing accuracy of the user. \\ 
    \hline
    FREQ13 & 2 & The system should display the typing speed of the user. \\ 
    \hline
    FREQ14 & 4 & The system shall end the typing lesson when all characters are completed and “ENTER” is pressed. \\ 
    \hline
    FREQ15 & 3 & The system should display the results (time, typing accuracy, and typing speed) of the lesson after the lesson is done. \\ 
    \hline
\end{longtable}
\end{center}



\section{Non-functional Requirements}

\subsection{Look and Feel Requirements}

\subsection{Usability and Humanity Requirements}

\subsection{Performance Requirements}

\subsection{Operational and Environmental Requirements}

\subsection{Maintainability and Support Requirements}

\subsection{Security Requirements}

\subsection{Cultural Requirements}

\subsection{Legal Requirements}

\subsection{Health and Safety Requirements}

This section is not in the original Volere template, but health and safety are
issues that should be considered for every engineering project.

\section{Project Issues}

\subsection{Open Issues}

\subsection{Off-the-Shelf Solutions}

\subsection{New Problems}

\subsection{Tasks}

\subsection{Migration to the New Product}

\subsection{Risks}

\subsection{Costs}

\subsection{User Documentation and Training}

\subsection{Waiting Room}

\subsection{Ideas for Solutions}

\bibliographystyle{plainnat}

\bibliography{SRS}

\newpage

\section{Appendix}

This section has been added to the Volere template.  This is where you can place
additional information.

\subsection{Symbolic Parameters}

The definition of the requirements will likely call for SYMBOLIC\_CONSTANTS.
Their values are defined in this section for easy maintenance.


\end{document}
