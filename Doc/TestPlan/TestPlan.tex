\documentclass[12pt, titlepage]{article}

\usepackage{booktabs}
\usepackage{tabularx}
\usepackage{hyperref}
\hypersetup{
    colorlinks,
    citecolor=black,
    filecolor=black,
    linkcolor=red,
    urlcolor=blue
}
\usepackage[round]{natbib}

\title{SE 3XA3: Test Plan\\ReTouch}

\author{Team \#7, ReTouchers
		\\ Abrar Attia - attiaa1
		\\ Susan Fayez - fayezs
		\\ Mediha Munim - munimm
}

\date{\today}

\begin{document}

\maketitle

\pagenumbering{roman}
\tableofcontents
\listoftables
\listoffigures

\begin{table}[bp]
\caption{\bf Revision History}
\begin{tabularx}{\textwidth}{p{3cm}p{2cm}X}
\toprule {\bf Date} & {\bf Version} & {\bf Notes}\\
\midrule
Date 27 & 1.0 & PoC Tests added\\
Date 2 & 1.1 & Notes\\
\bottomrule
\end{tabularx}
\end{table}

\newpage

\pagenumbering{arabic}

This document ...

\section{General Information}

\subsection{Purpose}

\subsection{Scope}

\subsection{Acronyms, Abbreviations, and Symbols}
	
\begin{table}[hbp]
\caption{\textbf{Table of Abbreviations}} \label{Table}

\begin{tabularx}{\textwidth}{p{3cm}X}
\toprule
\textbf{Abbreviation} & \textbf{Definition} \\
\midrule
Abbreviation1 & Definition1\\
Abbreviation2 & Definition2\\
\bottomrule
\end{tabularx}

\end{table}

\begin{table}[!htbp]
\caption{\textbf{Table of Definitions}} \label{Table}

\begin{tabularx}{\textwidth}{p{3cm}X}
\toprule
\textbf{Term} & \textbf{Definition}\\
\midrule
Term1 & Definition1\\
Term2 & Definition2\\
\bottomrule
\end{tabularx}

\end{table}	

\subsection{Overview of Document}

\section{Plan}
	
\subsection{Software Description}

\subsection{Test Team}

\subsection{Automated Testing Approach}

\subsection{Testing Tools}

\subsection{Testing Schedule}
		
See Gantt Chart at the following url ...

\section{System Test Description}
	
\subsection{Tests for Functional Requirements}

\subsubsection{Area of Testing1}
		
\paragraph{Title for Test}

\begin{enumerate}

\item{test-id1\\}

Type: Functional, Dynamic, Manual, Static etc.
					
Initial State: 
					
Input: 
					
Output: 
					
How test will be performed: 
					
\item{test-id2\\}

Type: Functional, Dynamic, Manual, Static etc.
					
Initial State: 
					
Input: 
					
Output: 
					
How test will be performed: 

\end{enumerate}

\subsubsection{Area of Testing2}

...

\subsection{Tests for Nonfunctional Requirements}

\subsubsection{Area of Testing1}
		
\paragraph{Title for Test}

\begin{enumerate}

\item{test-id1\\}

Type: 
					
Initial State: 
					
Input/Condition: 
					
Output/Result: 
					
How test will be performed: 
					
\item{test-id2\\}

Type: Functional, Dynamic, Manual, Static etc.
					
Initial State: 
					
Input: 
					
Output: 
					
How test will be performed: 

\end{enumerate}

\subsubsection{Area of Testing2}

...

\subsection{Traceability Between Test Cases and Requirements}

\section{Tests for Proof of Concept}

The Proof of Concept testing will focus on validating that all the identified risks in the development of the application can be overcome. The determined risks include running the code concurrently, such as running a constant timer alongside a main program that requests and waits for keyboard input. Another risk that needs to be resolved and tested includes 

\subsection{Area of Testing1}
		
\paragraph{Title for Test}

\begin{enumerate}

\item{test-id1\\}

Type: Functional, Dynamic, Manual, Static etc.
					
Initial State: 
					
Input: 
					
Output: 
					
How test will be performed: 
					
\item{test-id2\\}

Type: Functional, Dynamic, Manual, Static etc.
					
Initial State: 
					
Input: 
					
Output: 
					
How test will be performed: 

\end{enumerate}

\subsection{Area of Testing2}

...

	
\section{Comparison to Existing Implementation}	
There is currently no required tests to be done to compare the application to the existing implementation.
				
\section{Unit Testing Plan}

The JUnit unit testing framework will be used to implement the unit testing for the ReTouch application.
		
\subsection{Unit testing of internal functions}

When testing the internal functions of the application, the unit tests will be based on each methods input parameters and their expected versus derived outputs. Each method will be testing with standard expected inputs as well as exceptions to improve the programs robustness and reduce bugs. Additionally, every class in the program will include all the required import statements. Therefore, the unit tests will not require any stubs or drivers when testing individual methods or components that depend on multiple methods. 
		
\subsection{Unit testing of output files}

When testing the output files and program outputs, there will need to be a comparison between the required output and the generated output. However, at this time, there are no expected output files that will be created when the program is run. 		

\bibliographystyle{plainnat}

\bibliography{SRS}

\newpage

\section{Appendix}

This is where you can place additional information.

\subsection{Symbolic Parameters}

The definition of the test cases will call for SYMBOLIC\_CONSTANTS.
Their values are defined in this section for easy maintenance.

\subsection{Usability Survey Questions?}

This is a section that would be appropriate for some teams.

\end{document}